\documentclass{article}
\usepackage[utf8]{inputenc}
\usepackage{listings}
\usepackage{a4wide}
\usepackage{upgreek}
\usepackage{hyperref}
\usepackage{amssymb}
\usepackage{amsmath}
\usepackage{changepage}

\title{TMV027 Assignment 1 \\ Finite Tomato Theory and Formal Languages}
\author{Jakob Jarmar --- 900310-5617 [jarmar@student.chalmers.se]}
\date{}

\frenchspacing
\renewcommand\thesubsection{\thesection\alph{subsection}}

\usepackage[parfill]{parskip}

\begin{document}

\maketitle

\section{}
Proposition:
\begin{equation}
\begin{split}
& \forall n \in \mathbb{N}.\; 9^n - 1 \text{ is divisible by } 8 \\
\iff \; & \forall n \in \mathbb{N} . \exists k \in \mathbb{N}.\; 9^n - 1 = 8k
\end{split}
\end{equation}

Test test.

En ändring på github!

Proof by induction. Let $P(n) = \exists k \in \mathbb{N}.\; 9^n - 1 = 8k$.

En annan ändring på Sharelatex!

Base case: $P(0)$.
\begin{equation} \label{1_base}
\begin{split}
9^0 - 1 & = 8k \\
1 - 1 & = 8k \\
0 & = 8k \\
0 & = k
\end{split}
\end{equation}

Base case is OK. Inductive hypothesis is $P(n) \implies P(n+1)$:
\begin{equation}
\begin{split}
\exists k_n \in \mathbb{N} .\; 9^n - 1 = 8k_n \implies \exists k_{n+1} \in \mathbb{N}.\; 9^{n+1} - 1 = 8k_{n+1}
\end{split}
\end{equation}

Proof:
\begin{align}
9^n - 1 & = 8k_n && \text{premise of hypothesis} \\
\iff 9(9^n - 1) & = 9 \cdot 8k_n \\
\iff 9^{n+1} - 9 & = 8 \cdot 9k_n \\
\iff 9^{n+1} - 1 & = 8(9k_n + 1) && \text{let $k_{n+1} = 9k_n+1 \;\; (\in \mathbb{N})$} \\
\iff 9^{n+1} - 1 & = 8k_{n+1}
\end{align}

Thus the inductive hypothesis is proven. By the base case~(\ref{1_base}), the premise of the inductive hypothesis is true for $n = 0$, and so $P(0)$, $P(1)$, ... hold.

\newpage

\section{}
Computations:
\begin{align*}
f(1) &= f(0+1) &g(1) &= g(0+1) &h(1) &= h(0+1) \\
&= 2h(0) + g(0) + 4 &&= h(0) + 1 &&= g(0) + 3 \\
6&= 2 \cdot 1 + 0 + 4 &2&= 1 + 1 &3&= 0 + 3
\end{align*}
\begin{align*}
f(2) &= f(1+1) &g(2) &= g(1+1) &h(2) &= h(1+1) \\
&= 2h(1) + g(1) + 4 &&= h(1) + 1 &&= g(1) + 3 \\
12&= 2 \cdot 3 + 2 + 4 &4&= 3 + 1 &5&= 2 + 3
\end{align*}
\begin{align*}
f(3) &= f(2+1) &g(3) &= g(2+1) &h(3) &= h(2+1) \\
&= 2h(2) + g(2) + 4 &&= h(2) + 1 &&= g(2) + 3 \\
18&= 2 \cdot 5 + 4 + 4 &6&= 5 + 1 &7&= 4 + 3
\end{align*}

Proposition:
\begin{equation}
\forall n \in \mathbb{N}.\;f(n) = 3g(n)
\end{equation}
Let $P(n)$ be $f(n) = 3g(n)$. Furthermore, let $P'(n)$ be $P(n) \land h(n) = 1 + g(n)$. Clearly, $P'(n) \implies P(n)$. Therefore, we prove
\[\forall n \in \mathbb{N}.\;P'(n)\]
by mutual induction on $n$.

Base case, n = 0:
\begin{align*}
f(0) = 3g(0) &\land h(0) = 1 + g(0) \\
0 = 3\cdot 0 &\land 1 = 1 + 0
\end{align*}
Base case is OK.

Inductive hypothesis, $P'(n) \implies P'(n+1)$:
\[f(n) = 3g(n) \land h(n) = 1 + g(n) \implies f(n+1) = 3g(n+1) \land h(n+1) = 1 + g(n+1)\]
First: prove $f(n) = 3g(n) \land h(n) = 1 + g(n) \implies f(n+1) = 3g(n+1)$:
\begin{align*}
f(n+1) &= 2h(n) + g(n) + 4 && \text{by definition of } f(n+1) \\
f(n+1) &= 2h(n) + (h(n) - 1) + 4 && \text{by premise of IH} \\
f(n+1) &= 3h(n) + 3 \\
f(n+1) &= 3(g(n+1) - 1) + 3 &&  \text{by definition of }g(n+1) \\
f(n+1) &= 3g(n+1)
\end{align*}
Second: prove $f(n) = 3g(n) \land h(n) = 1 + g(n) \implies h(n+1) = 1 + g(n+1)$:
\begin{align*}
h(n+1) &= g(n) + 3 && \text{by definition of } h(n+1) \\
h(n+1) &= (h(n) - 1) + 3 && \text{by premise of IH} \\
h(n+1) &= h(n) + 2 \\
h(n+1) &= (g(n+1) - 1) + 2 && \text{by definition of } g(n+1) \\
h(n+1) &= 1 + g(n+1)
\end{align*}
Closure: thus the inductive hypothesis holds. As the base case $n = 0$ has been proven, the proposition has been proven for $n = 1,2,... \iff n \in \mathbb{N}$.
\newpage
\section{}
\subsection{}
Inductive definition of \texttt{Exp}.

Base cases:
\begin{itemize}
\item $i$ is an identifier $\implies i \in \texttt{Exp}$

(the specific meaning of \emph{identifier} is left undefined)
\end{itemize}
Inductive steps:
\begin{itemize}
\item Given $i_1,i_2 \in \texttt{Exp}$, then $i_1 + i_2 \in \texttt{Exp}$.
\item Given $i_1,i_2 \in \texttt{Exp}$, then $i_1 \cdot i_2 \in \texttt{Exp}$.
\end{itemize}
\subsection{}
Definition of $op$:
\begin{itemize}
\item $op\;i = 0$ ($i$ is an identifier)
\item $op\;e_1 + e_2 = 1 + op\;e_1 + op\;e_2$
\item $op\;e_1 \cdot e_2 = 1 + op\;e_1 + op\;e_2$
\end{itemize}
Definition of $id$:
\begin{itemize}
\item $id\;i = 1$ ($i$ is an identifier)
\item $id\;e_1 + e_2 = id\;e_1 + id\;e_2$
\item $id\;e_1 \cdot e_2 = id\;e_1 + id\;e_2$
\end{itemize}
\subsection{}
Proof by structural induction on the set of \texttt{Expr}s. Let $P(e)$ be $op(e) + 1 = id(e)$.

Base case:
For an identifier $i$, $op\;i = 0$ and $id\;i = 1$. Thus $P(i)$ holds.

Inductive steps:
\begin{itemize}
\item $e_1 + e_2$: we wish to prove that $P(e_1) \land P(e_2) \implies P(e_1 + e_2)$.
\begin{align*}
(op\;e_1+e_2) + 1 &= id\;e_1+e_2 && \text{our hypothesis} \\
(op\;e_1+e_2) + 1 &= id\;e_1 + id\;e_2 && \text{by definition} \\
(op\;e_1+e_2) + 1 &= op\;e_1 + op\;e_2 + 2 && \text{according to IH premise} \\
op\;e_1 + op\;e_2 + 2 &= op\;e_1 + op\;e_2 + 2 && \text{by definition}
\end{align*}
Thus we have proven that $P(e_1) \land P(e_2) \implies P(e_1 + e_2)$.
\item $e_1 \cdot e_2$: we wish to prove that
\begin{equation}
\label{skiten}
P(e_1) \land P(e_2) \implies P(e_1 \cdot e_2)
\end{equation}
Note that $op\;e_1+e_2 = op\;e_1\cdot e_2$ and that $id\;e_1+e_2 = id\;e_1 \cdot e_2$. This means that
\[ (op\;e_1+e_2)+1 = id\;e_1 + e_2 \iff (op\;e_1\cdot e_2)+1 = id\;e_1 \cdot e_2 \]
and therefore that $P(e_1+e_2) \iff P(e_1 \cdot e_2)$. Therefore, we can prove (\ref{skiten}) by showing that $P(e_1) \land P(e_2) \implies P(e_1 + e_2)$, which we already have done above.
\end{itemize}
Closure: $\forall e \in Expr. P(e)$.
\newpage

\section{}
Induction on the length of $x$. Let $P(xs)$ be $rev(rev(xs)) = xs$.

Base case: $\epsilon$
\begin{itemize}
\item
\begin{align*}
& rev(rev(\epsilon)) && \\
= &rev(\epsilon) && \text{by definition of $rev$} \\
= &\epsilon && \text{by definition of $rev$}
\end{align*}
\end{itemize}
Inductive step:
\begin{itemize}
\item Inductive hypothesis: $P(x) \implies P(xa)$ where $a \in \Sigma$ $(|a| = 1)$.
\begin{align*}
&rev(rev(xa)) && \\
=&rev(rev(a)rev(x)) && \text{given in assignment} \\
=&rev(rev(x))rev(a) && \text{by definition of $rev$} \\
=&x rev(a) && \text{by premise of IH} \\
=&x rev(a\epsilon) &&  \text{by definition of concatenation} \\
=&x rev(\epsilon) a && \text{by definition of rev} \\
=&x \epsilon a && \text{by definition of rev} \\
=&xa && \text{by definition of concatenation}
\end{align*}
\end{itemize}
Closure: Thus $P(\epsilon)$ and $\forall a \in \Sigma.P(x) \implies P(xa)$. Thus, $\forall x \in \Sigma.P(x)$.
\end{document}


